%% COMPILATION XeLaTeX

\documentclass[11pt,a4paper]{article}
\usepackage[left=2cm,right=2cm,top=4cm,bottom=2.5cm]{geometry}
\usepackage[english,french]{babel}

\usepackage[utf8]{inputenc}

%% PACKAGES
\usepackage[T1]{fontenc}
\usepackage{fancyhdr}
\usepackage{lastpage}
\usepackage{wallpaper}
\usepackage{sectsty}    %Pour paramétrer les titres

\usepackage{xcolor}
\definecolor{rougeECL}{cmyk}{0.08,1,0.70,0.32}

\usepackage{amsmath}
\usepackage{amsfonts}
\usepackage{amssymb}
\usepackage{mathrsfs}
%\usepackage{dsfont}
%\usepackage{stmaryrd} %\llbracket

\usepackage{textcomp}
\usepackage{textgreek} %lettres grecques avec \textalpha,...

\usepackage{graphicx}
\usepackage{float}

\usepackage{hyperref}
\usepackage[backend=biber,sorting = none]{biblatex}

\usepackage[autostyle = try]{csquotes}
\MakeOuterQuote{"}

%% COMMANDES COMMUNES
\def\sc{\textsc} \def\bf{\textbf} \def\it{\textit} \def\sl{\textsl} \def\tt{\texttt} \def\sf{\textsc}
\def\up{\textsuperscript} \def\down{\textsubscript}

%% COMMANDES MATHEMATIQUES
\everymath{\displaystyle}

\def\lp{\left(} \def\rp{\right)}
\def\lci{\left[} \def\rci{\right]} \def\lce{\left]} \def\rce{\right[} %Crochets
\def\la{\left\lbrace} \def\ra{\right\rbrace} %Accolades
\def\lps{\left\langle} \def\rps{\right\rangle} %Produit scalaire
\def\lb{\left|} \def\rb{\right|}

\def\eps{\varepsilon}
\def\dr{\partial}
%\def\iint{\int\!\!\!\!\int}
%\def\iiint{\int\!\!\!\!\int\!\!\!\!\int}

\renewcommand{\vec}{\overrightarrow}
\def\rm{\mathrm} \def\scr{\mathscr} \def\bb{\mathbb} \def\cal{\mathcal} \def\frak{\mathfrak} \def\tx{\text}


\newcommand{\appli}[4]{\begin{array}[t]{rcl} #1 & \to    & #2 \\ #3 & \mapsto &#4 \end{array}}   %  application

\DeclareMathOperator{\grad}{\vec{\rm{grad}}}\DeclareMathOperator{\rot}{\vec{\rm{rot}}}\DeclareMathOperator{\dive}{div}

%% ENVIRONNEMENTS
\newtheorem{rem}{Remarque}[section]

\usepackage{thmbox}

\newtheorem{theorem}[rem]{Théorème}
\newtheorem{boxrem}[rem]{Remarque}
\newtheorem{boxdef}[rem]{Définition}

\author{Premier \sc{Auteur} \and Deuxième \sc{Auteur} \and Troisième \sc{Auteur}}
%\date{Date du rapport} % Pour imposer la date sur la page de garde
\title{\vfill UE XXX\\\bf{Titre du document}}

%\addbibresource{bibliographie.bib}

\fancypagestyle{plain}{
\fancyhead{}
\fancyhead[C]{UE XXX}
\fancyhead[R]{Titre du rapport}
\fancyfoot{}
\fancyfoot[R]{\thepage/\pageref{LastPage}}
\def\headrulewidth{0pt}
}
\pagestyle{plain}

\CenterWallPaper{1}{logos/charte-graphique-ECL.pdf}

\usepackage{fontspec}	%Pour le choix des polices, compilation avec XeLaTeX obligatoirement
\setmainfont{Roboto}

\allsectionsfont{\color{rougeECL}\fontspec{Roboto}} %sectsty

\usepackage{lipsum} % La commande \lipsum[x-y] génère un texte type en pseudo-latin


%%%%% DOCUMENT %%%%%

\begin{document}
\begin{titlepage}
%\setlength{\topmargin}{6cm}
\maketitle
\thispagestyle{empty}
\vfill
\textbf{Année} 2019-2020\hfill\textbf{Enseignant:} Prénom \sc{Nom}
\end{titlepage}

~
\tableofcontents

\pagebreak

\part*{Introduction}\addcontentsline{toc}{part}{Introduction}
\lipsum[18]

\part{Titre} % On peut commencer directement par le niveau \section
\section{Section}
\subsection{Sous-section}
\subsubsection{Sous-sous-section}
\paragraph{Lipsum}\lipsum[1-3]
\subparagraph{Sous-paragraphe}\lipsum[4-10]

\subsection{Exemples}
\begin{figure} % Flottant : placé automatiquement
    \centering
    \includegraphics[width=16cm]{logos/logo-ecl-rectangle-quadri-print.jpg}
    \caption{Logo de l'\' Ecole Centrale de Lyon}
    \label{fig:logo_ecl}
\end{figure}

\begin{table}[H] % H impose l'emplacement de l'environnement flottant (table ou figure) ; p = page séparée réservée aux flottants ; t = top (haut de page) ; b = bottom ; \clearpage permet de vider le tampon des flottants (place tous les flottants restants apparus précédemment et non encore placés)
    \centering
    \begin{tabular}{|l||c|r|}
        \hline %ligne de séparation horizontale
        colonne 1 & colonne 2 & colonne 3\\
        \hline
        a & b & c\\
        \hline
    \end{tabular}
    \caption{Exemple de tableau}
    \label{tab:exemple}
\end{table}

\begin{rem}
\lipsum[16]
\end{rem}

\begin{theorem}
\lipsum[17]
\end{theorem}

\part*{Conclusion}\addcontentsline{toc}{part}{Conclusion}

\lipsum[11-15]

\addcontentsline{toc}{section}{Références}%\printbibliography

\listoffigures\addcontentsline{toc}{section}{Table des figures}
\listoftables\addcontentsline{toc}{section}{Liste des tableaux}

\appendix
\pagebreak
\part*{Annexes}\addcontentsline{toc}{part}{Annexes}
\section{Annexe 1}

\end{document}